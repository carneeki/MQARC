%% LyX 1.6.6.1 created this file.  For more info, see http://www.lyx.org/.
%% Do not edit unless you really know what you are doing.
\documentclass[a4paper,twoside,english]{IEEEtran}
\usepackage[T1]{fontenc}
\usepackage[latin9]{inputenc}

\makeatletter
%%%%%%%%%%%%%%%%%%%%%%%%%%%%%% User specified LaTeX commands.
\makeatother

\makeatother

\usepackage{babel}

\begin{document}
%\markboth{Macquarie University Amateur Radio Club}{Document Style Guide}%



\title{Rules of Association}


\author{Adam Carmichael (SID 41963539)\\
 {\normalsize Macquarie University NSW 2019 Australia}\\
 {\normalsize {} E-mail: adam.carmichael@students.mq.edu.au}\\
 {\normalsize {} $29^{th}$ June 2010. }}
\maketitle
\begin{abstract}
This document outlines the rules of association based on the Model
rules for incorporated associations under the Associations Incorporation
Act, 1984.

\tableofcontents{} 
\end{abstract}

\section{Preliminary}


\subsection{Definitions}
\begin{enumerate}
\item In this constitution: 

\begin{enumerate}
\item Director-general means the Director-General of the Department
of Services, Technology and Administration.
\item Ordinary committee member means a member of the committee
who is not an office-bearer of the association, as referred to in
rule 14(2)
\item Secretary means:

\begin{enumerate}
\item the person holding office under these rules as secretary of the association,
or
\item if no such person holds that office - the public officer of the association
\end{enumerate}
\item Special general meeting means a general meeting of the association
other than an annual general meeting
\item the Act means the Associations Incorporation Act 2009
\item the Regulation means the Assocations Incorporation Regulation 2010
\end{enumerate}
\item In this constitution:

\begin{enumerate}
\item a reference to a function includes a reference to a power, authority
and duty, and 
\item a reference to the exercise of a function includes, if the function
is a duty, a reference to the performance of the duty. 
\end{enumerate}
\item The provisions of the Interpretation Act 1987 apply to an in respect
of this constitution in the same manner as those provisions would
so apply if this constitution were an instrument made under the Act. 
\end{enumerate}

\section{Membership}


\subsection{Membership generally}
\begin{enumerate}
\item A person is eligible to be a member of the association if

\begin{enumerate}
\item the person is a natural person, and 
\item the person has been nominated and approved for membership of the association
in accordance with clause 3. 
\end{enumerate}
\item A person is taken to be a member of the association if

\begin{enumerate}
\item the person is a natural person and, 
\item the person was:

\begin{enumerate}
\item in the case of an unincorporated body that is registered as the association
- a member of that unincorporated body immediately before the registration
of the association, or 
\item in the case of an association that is amalgamated to form the relevant
association - a member of that other association immediately before
the amalgamation, or 
\item in the case of a registrable corporation that is registered as an
association - a member of the registrable corporation immediately
before that entity was registered as an association. 
\end{enumerate}
\item A person is taken to be a member of the association if the person
was one of the individuals on whose behalf an application for registration
of the association under section 6 (1) (a) of the Act was made. 
\item who has been approved for membership of the association by the committee
of the association, and 
\item is a student or employee of Macquarie University 
\end{enumerate}
\end{enumerate}

\subsection{Nomination for membership}
\begin{enumerate}
\item A nomination of a person for membership of the association:

\begin{enumerate}
\item must be made by a member of the association in writing in the form
set out in Appendix 1, or Appendix 3 to these rules, and 
\item must be lodged with the secretary of the association. 
\end{enumerate}
\item As soon as practicable after receiving a nomination for a membership,
the secretary must refer the nomination to the committee which is
to determine whether to approve or reject the nomination. 
\item As soon as practicable after the committee makes that determination,
the secretary must:

\begin{enumerate}
\item notify the nominee that the committee approved or rejected the nomination
(whichever is applicable), and 
\item if the committee approved the nomination, request the nominee to pay
(within the period of 28 days after receipt by the nominee of the
notification) the sum payable under these rules by a member as entrance
fee and annual subscription. 
\end{enumerate}
\item The secretary must, on payment by the nominee of the amounts referred
to in subclause (3) (b) within the period referred to in that provision,
enter or cause to be entered the nominee's name in the register of
members and, on the name being so entered, the nominee becomes a member
of the association. 
\end{enumerate}

\subsection{Cessation of membership}

A person ceases to be a member of the association if the person 
\begin{enumerate}
\item dies, or
\item resigns membership, or 
\item is expelled from the association 
\end{enumerate}

\subsection{Membership entitlements not transferable}

A right, privilege or obligation which a person has by reason of being
a member of the association:

\begin{enumerate}[a]
\item is not capable of being transferred or transmitted to another person,
and 
\item terminates on cessation of the person's membership 
\end{enumerate}

\subsection{Resignation of membership}
\begin{enumerate}
\item A member of the association is not entitled to resign that membership
except in accordance with this rule. 
\item A member of the association who has paid all amounts payable by the
member to the association of the members membership may resign from
membership of the association by first giving to the secretary written
notice of at least 1 week (or such other period as the committee may
determine) of the member's intention to resign and, on the expiration
of the period of notice, the member ceases to be a member. 
\item If a member of the association ceases to be a member under clause
(2), and in every other case where a member ceases to hold membership,
the secretary must make an appropriate entry in the register of members
recording the date on which the member ceased to be a member. 
\end{enumerate}

\subsection{Register of members}
\begin{enumerate}
\item The public officer of the assocation must establish and maintain a
register of members of the association specifying the name and address
of each person who is a member of the association toether with the
date on which the person became a member. 
\item The register of members must be kept at the principal place of administration
of the association and must be open for inspection, free of charge,
by an member of the association at any reasonable hour. 
\item A member of the association may obtain a copy of any part of the register
on payment of a fee of \$1 for each page copied, or if some other
amount is determined by the committee, that other amount. 
\end{enumerate}

\subsection{Fees and subscriptions}
\begin{enumerate}
\item A member of the association must, on admission to membership, pay
to the association a fee of \$1 or, if some other amount is determined
by the committee, that other amount. 
\item In addition to any amount payable by the member under clause (1),
a member of the association must pay to the association an annual
membership fee of \$2 or, if some other amount is determined by the
committee, that other amount:

\begin{enumerate}
\item except as provided by paragraph (b), before 1 July in each calendar
year, or 
\item if the member becomes a member on or after 1 July in any calendar
year - on becoming a member and before 1 July in each succeeding calendar
year. 
\end{enumerate}
\end{enumerate}

\subsection{Members' liabilities}

The liability of a member of the association to contribute towards
the payment of the debts and liabilities of the association or the
costs, charges and expenses of the winding up of the association is
limited to the amount, if any, unpaid by the member in respect of
membership of the assocation as required by rule 8.


\subsection{Resolution of internal disputes}
\begin{enumerate}
\item Disputes between members (in their capacity as members) of the association,
and disputes between members and the association, are to be referred
to a community justice centre for mediation in accordance with the
Community Justice Centres Act 1983. 
\item At least 7 days before a mediation session is to commence, the parties
are to exchange statements of the issues that are in dispute between
them and supply copies to the mediator. 
\end{enumerate}

\subsection{Disciplining of members}
\begin{enumerate}
\item A complaint may be made to the committee by any person that a member
of the association:

\begin{enumerate}
\item has persistently refused or neglected to comply with a provision or
provisions of these rules, or 
\item has persistently and wilfully acted in a manner prejudicial to the
interests of the association. 
\end{enumerate}
\item On receiving such a complaint, the committee:

\begin{enumerate}
\item must cause notice of the complaint to be served on the member concerned,
and 
\item must give the member at least 14 days from the time the notice is
served within which to make submissions to the committee in connection
with the complaint, and 
\item must take into consideration any submissions made by the member in
connection with the complaint. 
\end{enumerate}
\item The committee may, by resolution, expel the member from the association
or suspend the member from membership of the association if, after
considering the complaint and any submissions made in connection with
the complaint, it is satisfied that the facts alleged in the complaint
have been proved. 
\item If the committee expels or suspends a member, the secretary must,
within 7 das after the action is taken, cause written notice to be
given to the member of the action taken, of the reasons given by the
committee for having taken that action and of the member's right of
appeal under rule 12. 
\item The expulsion of suspension does not take effect:

\begin{enumerate}
\item until the expiration of the period within which the member is entitled
to appeal against the resolution concerned, or 
\item if within that period the member exercises the right of appeal, unless
and until the association confirms the resolution under rule 12(5)
whichever is the latter. 
\end{enumerate}
\end{enumerate}

\subsection{Right of appeal}
\begin{enumerate}
\item A member may appeal to the association in general meeting against
a aresolution of the committee under rule 11, within 7 days after
notice of the resolution is served on the member, by lodging with
the secretary a notice to that effect. 
\item The notice may, but need not, be accompanied by a statement of the
grounds on which the member intends to rely for the purposes of the
appeal. 
\item On receipt of a notice from a member under clause (1), the secretary
must notify the committee which is to convene a general meeting of
the association to be held within 28 days after the date on which
the secretary served the notice. 
\item At a general meeting of the association convened under clause (3):

\begin{enumerate}
\item no business other than the question of the appeal is to be transacted,
and 
\item the committee and the member must be given the opportunity to state
their respective cases orally or in writing, or both, and 
\item the members present are to vote by secret ballot on the question of
whether the resolution should be confirmed or revoked. 
\end{enumerate}
\item If at the general meeting the association passes a special resolution
in favour of the confirmation of the resolution, the resolution is
confirmed. 
\end{enumerate}

\section{The committee}


\subsection{Powers of the committee}

The committee is to be called the committee of management of the association
and, subject to the Act, the Regulation and these rules and to any
resolution passed by the association in general meeting: 
\begin{enumerate}
\item is to control and manage the affairs of the association, and 
\item may exercise all such functions as may be exercised by the association,
other than those functions that are required by these rules to be
exercise by a general meeting of members of the association, and 
\item has power to perform all such acts and do all such things as appear
to the committee to be necessary or desirable for the proper management
of the affairs of the association. 
\end{enumerate}

\subsection{Constitution and membership}
\begin{enumerate}
\item Subject in the case of the first members of the committee to section
21 of the Act, the committee is to consist of:

\begin{enumerate}
\item the office-bearers of the association, and 
\item two ordinary members, each of whom is to be elected at the annual
general meeting of the association under rule 15
\end{enumerate}
\item The office-bearers of the association are to be:

\begin{enumerate}
\item the president 
\item the treasurer, and 
\item the secretary. 
\end{enumerate}
\item Each member of the committee is, subject to these rules, to hold office
until the conclusion of the annual general meeting following the date
of the member's election, but is eligible for re-election. 
\item In the event of a casual vacancy occurring in the membership of the
committee, the committee may appoint a member of the association to
fill the vacancy and the member so appointed is to hold office, subject
to these rules, until the conclusion of the annual general meeting
next following the date of the appointment. 
\end{enumerate}

\subsection{Election of members}
\begin{enumerate}
\item Nominations of candidates for election as office-bearers of the association
or as ordinary members of the committee:

\begin{enumerate}
\item must be made in writing, signed by two members of the association
and accompanied by the written consent of the candidate (which may
be endorsed on the form of the nomination), and 
\item must be delivered to the secretary of he association at least 7 days
before the date fixed for the holding of the annual general meeting
at which the election is to take place. 
\end{enumerate}
\item If insufficient nominations are received to fill all vacancies on
the committee, the candidates nominated are taken to be elected and
further nominations are to be received at the annual general meeting. 
\item If insufficient further nominations are received, any vacant positions
remaining on the committee are taken to be casual vacancies. 
\item If the number of nominations received is equal to the number of vacancies
to be filled, the persons nominated are taken to be elected. 
\item If the number of nominations exceeds the number of vacancies to be
filled, a ballot is to be held. 
\item The ballot for the election of office-bearers and ordinary members
of the committee is to be conducted at the annual general meeting
in such usual and proper manner as the committee may direct. 
\end{enumerate}

\subsection{Secretary}
\begin{enumerate}
\item The secretary of the association must, as soon as practicable after
being appointed as secretary, lodge notice with the association of
his or her address. 
\item It is the duty of the secretary to keep minutes of

\begin{enumerate}
\item all appointments of office-bearers and members of the committee 
\item the names of members of the committee present at a committee meeting
or a general meeting, and 
\item all proceedings at committee meetings and general meetings 
\end{enumerate}
\item Minutes of proceedings at a meeting must be signed by the chairperson
of the meeting or by the chairperson of the next succeeding meeting. 
\end{enumerate}

\subsection{Treasurer}

It is the duty of the treasurer of the association to ensure: 
\begin{enumerate}
\item that all money due to the association is collected and received and
that all payments authorised by the association are made, and 
\item that correct books and accounts are kept showing the financial affairs
of the association, including full details of all receipts and expenditure
connected with the activities of the association 
\item all money receives is to be handled by U@MQ in accordance with the
rules of the Affiliation Agreement in effect. 
\end{enumerate}

\subsection{Casual vacancies}

For the purpose of these rules, a casual vacancy in the office of
a member of the committee occurs if the meber: 
\begin{enumerate}
\item dies, or 
\item ceases to be a member of the association, or 
\item becomes an insolvent under administration within the meaning of the
Corporations Act 2001 of the Commonwealth, or 
\item resigns office by notice in writing given to the secretary, or 
\item is removed from office under rule 19, or 
\item becomes a mentally incapacitated person, or 
\item is absent without the consent of the committee from all meetings of
the committee held during a period of 6 months. 
\end{enumerate}

\subsection{Removal of member}
\begin{enumerate}
\item The association in general meeting may by resolution remove any member
of the committee from the office of member before the expiration of
the member's term of office and may by resolution appoint another
person to hold office until the expiration of the term of office of
the member so removed. 
\item If a member of the committee to whom a proposed resolution referred
to in clause (1) relates makes representations in writing to the secretary
or president (not exceeding a reasonable length) and requests that
the representation be notified to the members of the association,
the secretary or the president may send a copy of the representations
to each member of the association or, if the representations are not
so sent, the member is entitled to require that the representations
be read out at the meeting at which the resolution is considered. 
\end{enumerate}

\subsection{Meetings and quorum}
\begin{enumerate}
\item The committee must meet at least three times in each period of 12
months at such place and time as the committee may determine. 
\item Additional meetings of the committee may be convened by the president
or by any member of the committee. 
\item Oral or written notice of a meeting of the committee must be given
by the secretary to each member of the committee at least 48 hours
(or such other period as may be unanimously agreed on by the members
of the committee) before the time appointed for the holding of the
meeting. 
\item Notice of a meeting given under clause (3) must specify the general
nature of the business to be transacted at the meeting and no business
other than that business is to be transacted at the meeting, except
business which the committee members present at the meeting unaninmously
agree to treat as urgent business. 
\item Any three members of the committee constitute a quorum for the transaction
of the business of a meeting of the committee. 
\item No business is to be transacted by the committee unless a quorum is
present and if, within half an hour of the time appointed for the
meeting, a quorum is not present, the meeting is to stand adjourned
to the same place and at the same hour of the same day in the following
week. 
\item If at the adjourned meeting a quorum is not present within half an
hour of the time appointed for the meeting, the meeting is to be rescheduled. 
\item At a meeting of the committee:

\begin{enumerate}
\item the president, or in the president's absence, the secretary is to
preside, or 
\item if the president and the secretary are absent or unwilling to act,
such one of the remaining members of the office-bearers or committee
as may be chosen by the members present at the meeting is to preside. 
\end{enumerate}
\end{enumerate}

\subsection{Delegation by committee to sub-committee}
\begin{enumerate}
\item The committee may, by instrument in writing, delegate to one or more
sub-committees (consisting of such member or members of the association
as the committee thinks fit) the exercise of such of the functions
of the committee as are specified in the instrumeent, other than:

\begin{enumerate}
\item this power of delegation, and 
\item a function which is a duty imposed on the committee by the Act or
by any ther law 
\end{enumerate}
\item A function the exercise of which has been delegated to a sub-committee
under this rule may, while the delegation remains unrevoked, be exercised
from time to time by the sub-committee in accordance with the terms
of the delegation. 
\item A delegation under this section may be made subject to such conditions
or limitations as to the exercise of an function, or as to time or
circumstances, as may be specified in the instrument of delegation. 
\item Despite any delegation under this rule, the committee may continue
to exercise any function delegated. 
\item Any act of thing done or suffered by a sub-committee acting int the
exercise of a delegation under this rule has the same force and effect
as it would have if it had been done or suffered by the committee. 
\item the committee may, by insturment in writing, revoke wholly or in part
any delegation under this rule. 
\item A sub-committee may meet and adjourne, as it thinks proper. 
\end{enumerate}

\subsection{Voting and decisions}
\begin{enumerate}
\item Questions arising at a meeting of the committee or of any sub-committee
appointed by the committee are to be determined by a majority of the
votes of members of the committee or sub-committee present at the
meeting. 
\item Each member present at a meeting of the committee or of any sub-committee
appointed by the committee (includin the person presiding at the meeting)
is entitled to one vote but, in the event of an equality of votes
on any question, the person presiding may exercise a second or casting
vote. 
\item Subject to rule 20(5), the committee may act despite any vacancy on
the committee. 
\item Any act or thing done or suffered, or purporting to have been done
or suffered, by the committee or by a sub-committee apointed by the
committee, is valid and effectual despite any defect that may afterwards
be discovered in the appointment or qualification of any member of
the committee of sub-committee. 
\end{enumerate}

\section{General meeting}


\subsection{Annual general meeting - holding of}
\begin{enumerate}
\item With the exception of the first annual general meeting of the association,
the association must, at least once in each calendar year and within
the period of 6 months after the expiration of each financial year
of the association, convene an annual general meeting of its members. 
\item The association must hold its first annual general meeting

\begin{enumerate}
\item within the period of 18 months after its incorporation under the Act,
and 
\item within the period of 6 months after the expiration of the first financial
year of the association. 
\end{enumerate}
\item Clauses (1) and (2) have effect subject to any extension or permission
granted by the Commissioner under section 26(3) of the Act. 
\end{enumerate}

\subsection{Annual general meeting - calling of and business at}
\begin{enumerate}
\item The annual general meeting of the association is, subject to the Act
and to rule 23, to be convened on such date and at such place and
time as the committee thinks fit. 
\item In addition to any other business which may be transacted at an annual
general meeting, the business of an annual general meeting is to include
the following:

\begin{enumerate}
\item to confirm the minutes of the last preceding annual general meeting
and of any special general meeting held since that meeting, 
\item to receive from the committee reports on the activities of the association
during the last preceding financial year, 
\item to elect office-bearers of the association and ordinary members of
hte committee, 
\item to receive and consider the statement which is required to be submitted
to members under section 26(6) of the Act. 
\end{enumerate}
\item An annual general meeting must be specified as such in the notice
convenening it. 
\end{enumerate}

\subsection{Special general meetings - calling of}
\begin{enumerate}
\item The committee may, whenever it thinks fit, convene a special general
meeting of the association. 
\item The committee must, on the requisition in writing of at least 5 percent
of the total number of members, convene a special general meeting
of the association. 
\item A requisition of members for a special general meeting:

\begin{enumerate}
\item must state the purpose or purposes of the meeting, and 
\item must be signed by the members making the requistion, and 
\item must be lodged with the secretary, and 
\item may consist of several documents in a similar form, each signed by
one or more of the members making the requisition. 
\end{enumerate}
\item If the committee fails to convene a special general meeting to be
held within 1 month after that date on which a requisition of members
for the meeting is lodged with the secretary, any one or more of the
members who made the requisition may convene a special general meeting
to be held not later than 3 months after that date. 
\item A special general meeting convened by a member or members as referred
to in clause (4) must be convened as nearly as is practicable in the
same manner as general meetings are convened by the committee and
any member who consequently incurs the expenses is entitled to be
reimbursed by the association for any expense so so incurred. 
\end{enumerate}

\subsection{Notice}
\begin{enumerate}
\item Except if the nature of the busienss proposed to be dealt with at
a general meeting requires a special resolution of the association,
the secretary must, at least 14 days before the date fixed for the
holding of the general meeting, give a notice to each member specifying,
place, date and time of the meeting and the nature of the business
proposed to be transacted at the meeting. 
\item If the nature of the business proposed to be dealt with at a general
meeting requires a special resolution of the association, the secretary
must, at least 21 days before the date fixed for the holding of the
general meeting, cause notice to be given to each member specifying,
in addition to the matter required under clause (1), the intention
to propose the resolution as a special resolution. 
\item No business other than that specified in the notice convening a general
meeting is to be transacted at the meeting except, in the case of
an annual general meeting, business which may be transacted under
rule 24(c). 
\item A member desiring to bring any business before a general meeting may
give notice in writing of that business to the secretary who must
include that business in the next notice calling a general meeting
given after receipt of the notice from the member. 
\end{enumerate}

\subsection{Procedure}
\begin{enumerate}
\item No item of business is to be transacted at a general meeting unless
a quorum of members entitled under these rules to vote is present
during the time the meeting is considering that item. 
\item Five members present in person (being members entitled under these
rules to vote at a general meeting) constitute a quorum for the transaction
of the business of a general meeting. item If within half an hour
after the appointed time for the commencement of a general meeting
a quorum is not present, the meeting:

\begin{enumerate}
\item if convened on the requisition of members, is to be dissolved, and 
\item in any other case, is to stand adjourned to the same day in the following
week at the same time and (unless another place is specified at the
time of the adjournment by the person presiding at the meeting or
communicated by written notice to members given before the day to
which the meeting is adjourned) at the same place. 
\end{enumerate}
\item If at the adjourned meeting a quorum is not present within half an
hour after the time appointed for the commencement of the meeting,
the members present (being at least 3) is to constitute a quorum. 
\end{enumerate}

\subsection{Presiding member}
\begin{enumerate}
\item The president or, in the president's absence, the secretary, is to
preside as chairperson at each general meeting of the association. 
\item If the president and the secretary are absent or unwilling to act,
the members present must elect one of their number to preside as chairperson
at the meeting. 
\end{enumerate}

\subsection{Adjournment}
\begin{enumerate}
\item The chairperson of a general meeting at which a quorum is present
may, with the consent of the majority of members present at the meeting,
adjourn the meeting from time to time and place to place, but no business
is to be transacted at an adjourned meeting other than the business
left unfinished at the meeting at which the adjournment took place. 
\item If a general meeting is adjourned for 14 days or more, the secretary
must give written or oral notice of the adjourned meeting to each
member of the association stating the place, date and time of the
meeting and the nature of the business to be transacted at the meeting. 
\item Except as provided in clauses (1) and (2), notice of an adjournment
of a general meeting or of the business to be transacted at an adjourned
meeting is not required to be given. 
\end{enumerate}

\subsection{Making of Decisions}
\begin{enumerate}
\item A question arising at a general meeting of the association is to be
determined on a show of hands and, unless before or on the declaration
of the show of hands a poll is demanded, a declaration by the chairperson
that a resolution has, on a show of hands, been carried or carried
unanimously or carried by a particular majority or lost, or an entry
to that effect in the minute book of the association, is evidence
of the fact without proof of the number or proportion of the votes
recorded in favour of or against that resolution. 
\item At a general meeting of the association, a poll may be demanded by
the chairperson or by at least 3 members present in person or by proxy
at the meeting. 
\item If a poll is demanded at a general meeting, the poll must be taken;

\begin{enumerate}
\item immediately in the case of a poll which relates to the election of
the chairperson of the meeting or to the question of an adjournment,
or 
\item in any other case, in such manner and at such time before the close
of the meeting as the chairperson directs, 
\item and the resolution of the poll on the matter is taken to be the resolution
of the meeting on that matter. 
\end{enumerate}
\end{enumerate}

\subsection{Special resolution}

A resolution of the association is a special resolution: 
\begin{enumerate}
\item if it is passed by a majority which comprises at least threequarters
of such members of the association as, being entitled under these
rules so to do, vote in person or by proxy at a general meeting of
which at least 21 days' written notice specifying the intention to
propose the resolution as a special resolution was given in accordance
with these rules, or 
\item where it is made to appear to the Director-General that it is not
practicable for the resolution to be passed in the manner specified
in paragraph (a) if the resolution is passed in a manner specified
by the Director-General. 
\end{enumerate}

\subsection{Voting}
\begin{enumerate}
\item On any question arising at a general meeting of the association a
member has one vote only. 
\item All votes must be given personally or by proxy but no member may hold
more than 5 proxies. 
\item In the case of an equality of votes on a question at a general meeting,
the chairperson of the meeting is entitled to exercise a second or
casting vote. 
\item A member or proxy is not entitled to vote at any general meeting of
the association unless all money due and payable by the member or
proxy to the association has been paid, other than the amount of the
annual subscription payable in respect of the then current year. 
\end{enumerate}

\subsection{Appointment of proxies}
\begin{enumerate}
\item Each member is to be entitled to appoint another member as proxy by
notice given to the secretary no later than 24 hours before the time
of the meeting in respect of which the proxy is appointed. 
\item The notice appointing the proxy is to be in the form set out in Appendix
2 to these rules. 
\end{enumerate}

\section{Miscellaneous}


\subsection{Insurance}
\begin{enumerate}
\item The association must effect and maintain insurance under section 44
of the Act. 
\item In addition to the insurance required under clause (1), the association
may effect and maintain other insurance. 
\end{enumerate}

\subsection{Funds - source}
\begin{enumerate}
\item The funds of the association are to be derived from entrance fees
and annual subscriptions of members, donations and, subject to any
resolution passed by the association in general meeting, such other
sources as the committee determines. 
\item All money received by the association must be deposited as soon as
practicable and without deduction to the credit of the association's
bank account. 
\item The association must, as soon as practicable after receiving any money,
issue an appropriate receipt. 
\end{enumerate}

\subsection{Funds - management}
\begin{enumerate}
\item Subject to any resolution passed by the association in general meeting,
the funds of the association are to be used in pursuance of the objects
of the association in such manner as the committee determines. 
\item All cheques, drafts, bills of exchange, promissory notes and other
negotiable instruments must be signed by any 2 members of the committee
or employees of the association, being members or employees authorised
to do so by the committee. 
\end{enumerate}

\subsection{Alteration of objects and rules}

The statement of objects and these rules may be altered, rescinded
or added to only by a special resolution of the association.


\subsection{Common seal}
\begin{enumerate}
\item The common seal of the association must be kept in the custody of
the public officer. 
\item The common seal must not be affixed to any instrument except by the
authority of the committee and the affixing of the common seal must
be attested by the signatures either of 2 members of the committee
or of 1 member of the committee and of the public officer or secretary. 
\end{enumerate}

\subsection{Custody of books}

Except as otherwise provided by these rules, the public officer must
keep in his or her custody or under his or her control all records,
books and other documents relating to the association.


\subsection{Inspection of books}

The records, books and other documents of the association must be
open to inspection, free of charge, by a member of the association
at any reasonable hour.


\subsection{Service of notices}
\begin{enumerate}
\item For the purpose of these rules, a notice may be served on or given
to a person:

\begin{enumerate}
\item by delivering it to the person personally, or 
\item by sending it by pre-paid post to the address of the person, or 
\item by sending it by facsimile transmission or some other form of electronic
transmission to an address specified by the person for giving or serving
the notice. 
\end{enumerate}
\item for the purpose of these rules, a notice is taken, unless the contrary
is proved, to have been given or served:

\begin{enumerate}
\item in the case of a notice given or served personally, on the date on
which it is received by the addressee, and 
\item in the case of a notice sent by pre-paid post, on the date when it
would have been delivered in the ordinary course of post, and 
\item in the case of a notice sent by facsimile transmission or some other
form of electronic transmission, on the date it was sent, or if the
machine from which the transmission was sent produces a report indicating
that the notice was sent on a later date, on that date. 
\end{enumerate}
\end{enumerate}

\section{Appendix 1}


\section{Appendix 2}


\section{Appendix 3}
\end{document}
